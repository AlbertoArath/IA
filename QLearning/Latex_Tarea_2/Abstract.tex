\documentclass{article}
\usepackage{graphicx}
\usepackage{amsmath, amssymb}
\usepackage{hyperref}

\title{Q-Learning with Neural Networks for Robotic Arm Control}
\author{Pablo Sánchez Orozco & Alberto Arath Figueroa Salomon}
\date{\today}

\begin{document}

\maketitle

\begin{abstract}
This report explores the use of Q-learning with neural networks (NN) for training a robotic arm model in a physics-accurate simulation. The objective is to demonstrate the advantages of using NN-based Q-learning over traditional Q-tables, particularly in high-dimensional robotic control problems. The Mujoco-based model and the Menagie physics engine were employed for accurate physics simulation.
\end{abstract}

\section{Introduction}
The intention of this project was to train a robotic arm using Q-learning with neural networks (NN) instead of a conventional Q-table. The complexity of the robot's state-action space necessitates a function approximator, which NN provides efficiently. 

\section{Methodology}
The robotic arm was simulated using Mujoco, with Menagie serving as the physics engine. The model consists of multiple articulated joints, each actuated by motors. The XML model features:
\begin{itemize}
    \item Six degrees of freedom: Base rotation, shoulder pitch, elbow movement, wrist pitch and roll, and jaw control.
    \item Defined friction loss and armature parameters for realistic motion.
    \item Motor constraints with position-based control.
    \item Contact modeling to simulate object interactions.
\end{itemize}

The Q-learning algorithm with NN was used to approximate the Q-value function, enabling generalization over large state-action spaces.

\section{Mathematical Formulation}
The Q-learning update rule is given by:
\begin{equation}
    Q(s, a) \leftarrow Q(s, a) + \alpha \left[ r + \gamma \max_{a'} Q(s', a') - Q(s, a) \right]
\end{equation}
where:
\begin{itemize}
    \item $Q(s, a)$ is the Q-value for state $s$ and action $a$.
    \item $\alpha$ is the learning rate.
    \item $\gamma$ is the discount factor.
    \item $r$ is the received reward.
    \item $s'$ is the next state.
\end{itemize}
Instead of storing $Q(s,a)$ in a table, we approximate it using a neural network:
\begin{equation}
    Q(s, a; \theta) \approx f(s, a; \theta)
\end{equation}
where $\theta$ represents the NN parameters trained via backpropagation.

\section{Why Use NN Instead of Q-table?}
A traditional Q-table is infeasible for a high-dimensional state space as it requires an entry for each state-action pair. Given the continuous action space of the robotic arm, NN provides a better alternative by enabling function approximation and generalization, reducing memory requirements and training time.

\section{Results and Discussion}
Our Mujoco-based robotic arm demonstrated improved convergence rates when using Q-learning with NN compared to discrete Q-tables. The model successfully executed complex motion sequences by leveraging neural network-based value estimation.

\section{Conclusion}
This report presented an implementation of Q-learning with neural networks for robotic arm control. The Mujoco and Menagie-based setup allowed for realistic physics modeling, and NN-based Q-learning proved effective in handling high-dimensional control tasks. Future work can explore reinforcement learning variants such as DDPG or PPO for further improvements.

\section{References}
Include references to reinforcement learning and Mujoco documentation.

\end{document}
